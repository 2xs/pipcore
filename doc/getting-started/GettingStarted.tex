\documentclass[10pt,a4paper,titlepage]{refart}

\usepackage[utf8x]{inputenc}
\usepackage{ucs}
\usepackage[english]{babel}
\usepackage{amsmath}
\usepackage{amsfonts}
\usepackage{amssymb}
\usepackage{listings}

\title{Get Started with Pip}

\lstset{
basicstyle=\ttfamily,
frame=single
}

\begin{document}

\maketitle

\tableofcontents

\pagebreak

\section{Setting up your development environment}

To get started with Pip, it is required to install the appropriate development
environment. This section describes the tools required by Pip as well as the
three ways to obtain a functional development environment.

\subsection{Required tools}

\index{coq} \marginlabel{Coq Proof Assistant:}
Pip's source code and formal proof of its memory isolation properties are
written using the Coq proof assistant. In order to compile Coq files and
generate the required intermediate files for the kernel to build, you will need
the 8.13.1 version of Coq. A proper way to install Coq is via opam.

\index{doxygen} \marginlabel{Doxygen:}
Pip's documentation is generated through CoqDoc (included with Coq) for the Coq
part, and Doxygen for the C part. The documentation is not mandatory to compile
Pip, but it is highly required that you compile it and keep it somewhere safe so
you always have some reference to read if you need some information about Pip's
internals.

\index{gcc} \marginlabel{GNU C Compiler:}
GCC is the only C compiler known to compile Pip correctly. CLANG, for example,
is not yet supported. To that end, you need a version of GCC capable of
producing 32bits ELF binaries.

\index{gdb} \marginlabel{GNU Debugger:}
The GNU Debugger allows you to debug a partition while it is executed on the
top of Pip. This is very useful during the development process. That's the
reason why GDB is not mandatory but highly recommended.

\index{gnu} \marginlabel{GNU Make:}
Althought Pip is known to compile on FreeBSD and OSX hosts, these need a GNU
software in order to perform the compilation, which is GNU Make.

\index{grub-mkrescue} \marginlabel{GNU GRUB:}
GNU GRUB is a boot loader which allows to create bootable ISO file. It is not
mandatory but required if you want to produce a bootable ISO file of your
project.

\index{nasm} \marginlabel{Netwide Assembler:}
Pip's assembly sources are compiled using the Netwide Assembler (NASM). A known
working version is version 2.14, although any version since 2.0 should be
working.

\index{opam} \marginlabel{OCaml Package Manager:}
Opam is the package manager for the OCaml programming language, the language in
which Coq is implemented. This is the proper way to install and pin the Coq
Proof Assistant to a specific version.

\index{pdflatex} \marginlabel{TeX Live}
TeX Live is an open source TeX distribution required to generate the
\textit{getting started} of Pip. It not mandatory, but required if you want to
generate this document.

\index{qemu} \marginlabel{QEMU:}
Although it is not required to build Pip, it is highly recommended to run Pip on
emulated hardware rather than physical hardware during development. As such,
QEMU is a known, multi-platform emulator, and is fully integrated into Pip's
toolchain.

\index{stack} \marginlabel{Haskell Stack:}
Pip uses a home-made extractor to convert Coq code into C code. In order to
compile this Extractor, which is written in Haskell, we use the Stack tool to
download and install automatically the required GHC and libraries.

\subsection{Virtual machine image}

This section describes step-by-step how to get a development environment from a
virtual machine image.

Before starting, you need to install a virtualization software such as
VirtualBox or VMware and download the virtual machine image. Once the download
is complete, you need to import the image into the virtualization software, then
start the virtual machine.

The login credentials are:

\begin{lstlisting}
Login: pip
Password: pip
\end{lstlisting}

or

\begin{lstlisting}
Login: root
Password: pip
\end{lstlisting}

Your development environment is ready.

\subsection{Docker image}

This section describes step-by-step how to get a development environment from
the Docker image of Pip. Before starting, you need to install Docker on your
machine and download the Docker image.

You can either run a new container from the Pip Docker image in interactive
mode with:

\begin{lstlisting}[language=bash]
    # Run Pip's image inside of a new container
    $ docker run -it --name pip pip bash

    # Run a command in the running container
    $ whoami

    # Exit the shell
    $ exit
\end{lstlisting}

or in detached interactive mode with:

\begin{lstlisting}[language=bash]
    # Run Pip's image inside of a new container
    $ docker run -dit --name pip pip bash

    # Run a command in the running container
    $ docker exec pip whoami
\end{lstlisting}

Once you are done with the container, you can stop it and remove it with:

\begin{lstlisting}[language=bash]
    # Stop the container
    $ docker stop pip

    # Remove the container
    $ docker rm pip
\end{lstlisting}

Before removing the container, make sure that you have saved all your changes:
any unsaved changes will be lost.

Your development environment is ready.

\subsection{Step-by-step installation}

This section describes step-by-step how to get a development environment on your
host machine. We assume that your machine is running a Debian-based Linux
distribution.

\subsubsection{Installing the required packages}

Update the apt package index:

\begin{lstlisting}[language=bash]
    $ sudo apt update
\end{lstlisting}

Install the required packages:

\begin{lstlisting}[language=bash]
    $ sudo apt install \
        build-essential \
        doxygen \
        gdb \
        git \
        grub2-common \
        grub-pc \
        haskell-stack \
        nasm \
        opam \
        qemu-system-i386 \
        texlive-base \
        xorriso
\end{lstlisting}

Download the GHC compiler if necessary in the \texttt{\$HOME/.stack}:

\begin{lstlisting}[language=bash]
    $ stack setup
\end{lstlisting}

Initialize the internal state of opam in the \texttt{\$HOME/.opam} directory:

\begin{lstlisting}[language=bash]
    $ opam init
    $ eval $(opam env)
\end{lstlisting}

Build Coq from source with opam:

\begin{lstlisting}[language=bash]
    $ opam pin add coq 8.13.1
\end{lstlisting}

\subsubsection{Getting source code}

Clone the git repository of the projects:

\begin{lstlisting}[language=bash]
    $ git clone ...
    $ git clone ...
\end{lstlisting}

\subsubsection{Building LibPip}

In order to build a partition, you will probably need to build LibPip:

\begin{lstlisting}[language=bash]
    # Build the libpip library
    $ make -C /path/to/libpip
\end{lstlisting}

Remember the path to the repository.

\subsubsection{Building Digger}

Build Digger through the stack tool:

\begin{lstlisting}[language=bash]
    # Initialize the submodules of pipcore
    $ git -C /path/to/pipcore submodule init

    # Update the submodules of pipcore
    $ git -C /path/to/pipcore submodule update

    # Compile Digger
    $ make -C /path/to/pipcore/tools/digger
\end{lstlisting}

\subsubsection{Partition toolchain configuration}

Configure the partition building toolchain:

\begin{itemize}
\item Open a terminal emulator
\item Go to Pip's repository
\item Navigate to the \texttt{src/arch/x86\_multiboot/partitions} directory
\item Copy the \texttt{toolchain.mk.template} file into \texttt{toolchain.mk}
\item Open \texttt{toolchain.mk} with a text editor
\end{itemize}

Here, you need to specify which compiler and LibPip to use. Basically, we will
use GCC as C compiler and assembler, and LD as linker. Set LIBPIP to
\texttt{/path/to/libpip}, which we defined previously as the path to LibPip's
repository. Feel free to replace those with your favourite toolchain (e.g.
i386-elf-gcc on Mac OSX):

\begin{lstlisting}[language=bash]
CC=gcc
LD=ld
AS=gcc

LIBPIP=/path/to/libpip
\end{lstlisting}

Your development environment is ready.

\section{Testing your development environment}

This section describes how to test your development environment, whether it is
from a virtual machine image, a Docker image or your host machine.

\subsection{Building the minimal partition}

Before to build pipcore, you need to build the minimal partition:

\begin{lstlisting}[language=bash]
    $ make -C \
    /path/to/pipcore/src/arch/
    x86_multiboot/partitions/minimal
\end{lstlisting}

\subsection{Building pipcore}

Now that you have built the minimal partition, you can build pipcore with the
partition on top of it:

\begin{lstlisting}[language=bash]
    $ make -C /path/to/pipcore
\end{lstlisting}

If it is the first time you run \texttt{make} command, it will execute the
\texttt{configure.sh} script. Follow the instructions on the screen to generate
the \texttt{toolchain.mk} file needed for the compilation.


You should find in \texttt{/path/to/pipcore} the ELF binary and an ISO image of
Pip.

\subsection{Testing in QEMU}

Now, you can run the ELF version of Pip in QEMU with:

\begin{lstlisting}[language=bash]
    $ make -C /path/to/pipcore qemu-elf
\end{lstlisting}

or run the ISO version with:

\begin{lstlisting}[language=bash]
    $ make -C /path/to/pipcore qemu-iso
\end{lstlisting}

This should display ``Hello world!'' on the serial link after a few seconds.

\end{document}
