%%%%%%%%%%%%%%%%%%%%%%%%%%%%%%%%%%%%%%%%%%%%%%%%%%%%%%%%%%%%%%%%%%%%%%%%%%%%%%%
%  © Université de Lille, The Pip Development Team (2015-2021)                %
%                                                                             %
%  This software is a computer program whose purpose is to run a minimal,     %
%  hypervisor relying on proven properties such as memory isolation.          %
%                                                                             %
%  This software is governed by the CeCILL license under French law and       %
%  abiding by the rules of distribution of free software.  You can  use,      %
%  modify and/ or redistribute the software under the terms of the CeCILL     %
%  license as circulated by CEA, CNRS and INRIA at the following URL          %
%  "http://www.cecill.info".                                                  %
%                                                                             %
%  As a counterpart to the access to the source code and  rights to copy,     %
%  modify and redistribute granted by the license, users are provided only    %
%  with a limited warranty  and the software's author,  the holder of the     %
%  economic rights,  and the successive licensors  have only  limited         %
%  liability.                                                                 %
%                                                                             %
%  In this respect, the user's attention is drawn to the risks associated     %
%  with loading,  using,  modifying and/or developing or reproducing the      %
%  software by the user in light of its specific status of free software,     %
%  that may mean  that it is complicated to manipulate,  and  that  also      %
%  therefore means  that it is reserved for developers  and  experienced      %
%  professionals having in-depth computer knowledge. Users are therefore      %
%  encouraged to load and test the software's suitability as regards their    %
%  requirements in conditions enabling the security of their systems and/or   %
%  data to be ensured and,  more generally, to use and operate it in the      %
%  same conditions as regards security.                                       %
%                                                                             %
%  The fact that you are presently reading this means that you have had       %
%  knowledge of the CeCILL license and that you accept its terms.             %
%%%%%%%%%%%%%%%%%%%%%%%%%%%%%%%%%%%%%%%%%%%%%%%%%%%%%%%%%%%%%%%%%%%%%%%%%%%%%%%

\documentclass[10pt,a4paper,titlepage]{refart}

\usepackage[utf8x]{inputenc}
\usepackage{ucs}
\usepackage[english]{babel}
\usepackage{amsmath}
\usepackage{amsfonts}
\usepackage{amssymb}
\usepackage{listings}
\usepackage{xcolor}
\usepackage{hyperref}

\title{Get Started with Pip}

\definecolor{commentcolor}{rgb}{0,0.6,0}
\definecolor{numbercolor}{rgb}{0.5,0.5,0.5}
\definecolor{stringcolor}{rgb}{0.58,0,0.82}
\definecolor{backgroundcolour}{rgb}{0.95,0.95,0.92}

\lstset {
    backgroundcolor=\color{backgroundcolour},
    basicstyle=\footnotesize,
    breakatwhitespace=false,
    breaklines=true,
    keepspaces=true,
    showspaces=false,
    showstringspaces=false,
    showtabs=false,
    tabsize=4,
    frame=single
}

\lstdefinestyle{BashStyle} {
    commentstyle=\color{commentcolor},
    keywordstyle=\color{black},
    stringstyle=\color{stringcolor},
    language=bash
}

\lstdefinestyle{CStyle} {
    emph={uint32_t},
    emphstyle={\color{magenta}},
    commentstyle=\color{commentcolor},
    keywordstyle=\color{magenta},
    stringstyle=\color{stringcolor},
    language=C
}

\hypersetup{
    colorlinks,
    citecolor=black,
    filecolor=black,
    linkcolor=black,
    urlcolor=black
}

\begin{document}

\maketitle

\tableofcontents

\pagebreak

\section{Setting up your development environment}

To get started with Pip, it is required to install the appropriate development
environment. This section describes the tools required by Pip as well as the
three ways to obtain a functional development environment.

\subsection{Required tools}

\index{coq} \marginlabel{Coq Proof Assistant:}
Pip's source code and formal proof of its memory isolation properties are
written using the Coq proof assistant. In order to compile Coq files and
generate the required intermediate files for the kernel to build, you will need
the 8.13.1 version of Coq. A proper way to install Coq is via opam.

\index{doxygen} \marginlabel{Doxygen:}
Pip's documentation is generated through CoqDoc (included with Coq) for the Coq
part, and Doxygen for the C part. The documentation is not mandatory to compile
Pip, but it is highly required that you compile it and keep it somewhere safe so
you always have some reference to read if you need some information about Pip's
internals.

\index{gcc} \marginlabel{GNU C Compiler:}
GCC is the only C compiler known to compile Pip correctly. CLANG, for example,
is not yet supported. To that end, you need a version of GCC capable of
producing 32bits ELF binaries.

\index{gdb} \marginlabel{GNU Debugger:}
The GNU Debugger allows you to debug a partition while it is executed on the
top of Pip. This is very useful during the development process. That's the
reason why GDB is not mandatory but highly recommended.

\index{gnu} \marginlabel{GNU Make:}
Althought Pip is known to compile on FreeBSD and OSX hosts, these need a GNU
software in order to perform the compilation, which is GNU Make 4.3 and above.

\index{grub-mkrescue} \marginlabel{GNU GRUB:}
GNU GRUB is a boot loader which allows to create bootable ISO file. It is not
mandatory but required if you want to produce a bootable ISO file of your
project.

\index{nasm} \marginlabel{Netwide Assembler:}
Pip's assembly sources for the x86 architecture are assembled using the Netwide
Assembler (NASM). A known working version is version 2.14, although any version
since 2.0 should be working.

\index{opam} \marginlabel{OCaml Package Manager:}
Opam is the package manager for the OCaml programming language, the language in
which Coq is implemented. This is the proper way to install and pin the Coq
Proof Assistant to a specific version.

\index{pdflatex} \marginlabel{TeX Live}
TeX Live is an open source TeX distribution required to generate the
\textit{getting started} of Pip. It not mandatory, but required if you want to
generate this document.

\index{qemu} \marginlabel{QEMU:}
Although it is not required to build Pip, it is highly recommended to run Pip on
emulated hardware rather than physical hardware during development. As such,
QEMU is a known, multi-platform emulator, and is fully integrated into Pip's
toolchain.

\index{stack} \marginlabel{Haskell Stack:}
Pip uses a home-made extractor to convert Coq code into C code. In order to
compile this Extractor, which is written in Haskell, we use the Stack tool to
download and install automatically the required GHC and libraries.

\subsection{Virtual machine image}

Before starting, you need to install a virtualization software such as
VirtualBox or VMware. You can follow the procedure on their websites. Once the
installation is completed, you need to download the archived OVA image of the
virtual machine and the SHA-256 message digest:

\begin{lstlisting}[style=BashStyle]
# Download the archived OVA image of the virtual machine
$ wget http://pip.univ-lille1.fr/image/vm/pip.tar.gz

# Download the SHA-256 message digest of the archived image
$ wget http://pip.univ-lille1.fr/image/vm/pip.tar.gz.sha256sum
\end{lstlisting}

When the download is complete, you can check the integrity of the downloaded
file:

\begin{lstlisting}[style=BashStyle]
$ sha256sum -c pip.tar.gz.sha256sum
\end{lstlisting}

Now, you have to extract the archived image:

\begin{lstlisting}[style=BashStyle]
$ tar -xvf pip.tar.gz
\end{lstlisting}

Once the extraction is complete, you have to import the OVA image into the
virtualization software, then start the virtual machine.

The login credentials are:

\begin{lstlisting}
Login: pip
Password: pip
\end{lstlisting}

or

\begin{lstlisting}
Login: root
Password: pip
\end{lstlisting}

Your development environment is ready.

\subsection{Docker image}

Before starting, you need to install Docker on your machine. You can follow the
procedure on their website. Once the installation is completed, you have to
download the archived Docker image and the SHA-256 message digest:

\begin{lstlisting}[style=BashStyle]
# Download the archived Docker image of Pip
$ wget http://pip.univ-lille1.fr/image/docker/pip.tar.gz

# Download the SHA-256 message digest of the archived image
$ wget http://pip.univ-lille1.fr/image/docker/pip.tar.gz.sha256sum
\end{lstlisting}

When the download is complete, you can check the integrity of the downloaded
file:

\begin{lstlisting}[style=BashStyle]
$ sha256sum -c pip.tar.gz.sha256sum
\end{lstlisting}

Now, you need to import the archived image:

\begin{lstlisting}[style=BashStyle]
$ docker load -i pip.tar.gz
\end{lstlisting}

and check that it is imported:

\begin{lstlisting}[style=BashStyle]
$ docker image ls
\end{lstlisting}

Once the Docker image imported, you can either run a new container from the
image in interactive mode:

\begin{lstlisting}[style=BashStyle]
    # Run Pip's image inside of a new container
    $ docker run -it --name pip pip bash

    # Run a command in the running container
    $ whoami

    # Exit the shell
    $ exit
\end{lstlisting}

or in detached mode:

\begin{lstlisting}[style=BashStyle]
    # Run Pip's image inside of a new container
    $ docker run -dit --name pip pip bash

    # Run a command in the running container
    $ docker exec pip whoami
\end{lstlisting}

When you are done with the container, you can stop it and remove it:

\begin{lstlisting}[style=BashStyle]
    # Stop the container
    $ docker stop pip

    # Remove the container
    $ docker rm pip
\end{lstlisting}

Before removing the container, make sure that you have saved all your changes:
any unsaved changes will be lost.

Your development environment is ready.

\subsection{Step-by-step installation}

This section describes step-by-step how to get a development environment on your
host machine. We assume that your machine is running a Debian-based Linux
distribution.

\subsubsection{Installing the required packages}

Update the apt package index:

\begin{lstlisting}[style=BashStyle]
    $ sudo apt update
\end{lstlisting}

For the x86 architecture, install the following necessary packages:

\begin{lstlisting}[style=BashStyle]
    $ sudo apt install build-essential doxygen gdb git grub2-common grub-pc haskell-stack nasm opam qemu-system-i386 texlive texlive-latex-extra xorriso
\end{lstlisting}

For the ARMv7 architecture, install the following necessary packages:

\begin{lstlisting}[style=BashStyle]
    $ sudo apt install build-essential doxygen gcc-arm-none-eabi gdb-multiarch git grub2-common grub-pc haskell-stack opam qemu-system-arm texlive texlive-latex-extra xorriso
\end{lstlisting}

Download the GHC compiler if necessary in the \texttt{\$HOME/.stack}:

\begin{lstlisting}[style=BashStyle]
    $ stack setup
\end{lstlisting}

Initialize the internal state of opam in the \texttt{\$HOME/.opam} directory:

\begin{lstlisting}[style=BashStyle]
    $ opam init
    $ eval $(opam env)
\end{lstlisting}

Build Coq from source with \texttt{opam}:

\begin{lstlisting}[style=BashStyle]
    $ opam pin add coq 8.13.1
\end{lstlisting}

\subsubsection{Getting source code}

First, you have to clone the \texttt{pipcore} repository which contains the
kernel, proof and documentation of Pip:

\begin{lstlisting}[style=BashStyle]
    $ git clone https://github.com/2xs/pipcore.git
\end{lstlisting}

Then, you may need the source code of the userland library of Pip, called
LibPip, which provides useful functions for calling the API or managing the data
structures of Pip:

\begin{lstlisting}[style=BashStyle]
    $ git clone https://github.com/2xs/libpip.git
\end{lstlisting}

\subsubsection{Building LibPip}

To build a partition on top of Pip, you will probably need LibPip.

To build Libpip for the x86 architecture:

\begin{lstlisting}[style=BashStyle]
    $ make -C /path/to/libpip ARCH=x86
\end{lstlisting}

To build LibPip for the ARMv7 architecture:

\begin{lstlisting}[style=BashStyle]
    $ make -C /path/to/libpip ARCH=armv7
\end{lstlisting}

\subsubsection{Building Digger}

In order to convert the Coq code into C code, you need to build the extractor,
called Digger. The first step is to download the source code:

\begin{lstlisting}[style=BashStyle]
    # Initialize your local configuration file
    $ git -C /path/to/pipcore submodule init

    # Fetch all the data from the digger project
    $ git -C /path/to/pipcore submodule update --remote
\end{lstlisting}

Then, build Digger through the stack tool:

\begin{lstlisting}[style=BashStyle]
    $ make -C /path/to/pipcore/tools/digger
\end{lstlisting}

\subsubsection{Configuration script}

The purpose of the configuration script is to detect whether the tools needed to
compile the project are installed. This script expects three mandatory
arguments: the target architecture, the name of the root partition and the path
to the LibPip. Optional arguments can also be provided. For more information:

\begin{lstlisting}[style=BashStyle]
./path/to/pipcore/configure.sh --help
\end{lstlisting}

To configure the project for the x86 architecture and the minimal root
partition:

\begin{lstlisting}[style=BashStyle]
./path/to/pipcore/configure.sh \
    --architecture=x86 \
    --partition-name=minimal \
    --libpip=/path/to/libpip
\end{lstlisting}

To configure the project for the ARMv7 architecture and the minimal root
partition:

\begin{lstlisting}[style=BashStyle]
./path/to/pipcore/configure.sh \
    --architecture=armv7 \
    --partition-name=minimal \
    --libpip=/path/to/libpip
\end{lstlisting}

Your development environment is ready.

\section{Testing your development environment}

This section describes how to test your development environment, whether it is
from a virtual machine image, a Docker image or your host machine.

\subsection{Building pipcore}

You can build pipcore with the root partition on top of it:

\begin{lstlisting}[style=BashStyle]
    $ make -C /path/to/pipcore
\end{lstlisting}

You should find in \texttt{/path/to/pipcore} directory the ELF binary and the
ISO image of Pip.

\subsection{Testing in QEMU}

You can test the ELF version of Pip in QEMU:

\begin{lstlisting}[style=BashStyle]
    $ make -C /path/to/pipcore qemu-elf
\end{lstlisting}

or test the ISO version:

\begin{lstlisting}[style=BashStyle]
    $ make -C /path/to/pipcore qemu-iso
\end{lstlisting}

This should display ``Hello world!'' on the serial link after a few seconds.

\section{User Guide}

\subsection{The minimal partition}

The purpose of the minimal partition is to show how to make a functional minimal
partition that prints ``Hello World'' on the serial link without the LibPip. To
go into details, see the source code of the minimal partition.

\subsubsection{Calling the API of Pip}

In order to keep the minimal partition as minimal as possible, we will not use
the LibPip library, but rather call the Pip API directly using inline assembly.

Before writing a character on the serial link, it is necessary to check if it is
ready to transmit. We must therefore write a function that allows us to retrieve
the state of the transmitting cycle of the serial link contained in the Line
Status Register (LSR). This register is accessible in read mode at address
\texttt{0x3FD} (\texttt{SERIAL\_PORT+5}). Since we are in \textit{userland}, we
cannot directly read the IO port using the \texttt{IN} instruction. We will have
to call the corresponding Pip service which is located at index \texttt{0x38}
in the Global Descriptor Table (GDT).

The function of the minimal partition that call the \texttt{IN} service of Pip
to retrieves the state of the transmitting cycle is the following:

\begin{lstlisting}[style=CStyle]
uint32_t serial_transmit_ready(void) {
    register uint32_t result asm("eax");
    asm (
       "push %1;"
       "lcall $0x38,$0x0;"
       "add $0x4, %%esp;"
       /* Outputs */
       : "=r" (result)
       /* Inputs */
       : "i" (SERIAL_PORT+5)
       /* Clobbers */
       :
    );
    return result & 0x20;
}
\end{lstlisting}

This Pip service expects to return the value read on the IO port in the
\texttt{EAX} register of the CPU. We therefore declare a variable that will be
stored in this register:

\begin{lstlisting}[style=CStyle]
register uint32_t result asm("eax");
\end{lstlisting}

It also expects to have one argument on the stack, which is the address of the
IO port to read. So we push on the stack the argument \texttt{\%1}, which is the
\texttt{SERIAL\_PORT+5} argument present as input operand:

\begin{lstlisting}
push %1;
\end{lstlisting}

Now that we have a variable stored in the \texttt{EAX} register and pushed the
argument onto the stack, we can make our far call:

\begin{lstlisting}
lcall $0x38,$0x0;
\end{lstlisting}

We clear the stack after the far call by adding \texttt{4} to the \texttt{ESP}
register:

\begin{lstlisting}
add $0x4, %%esp;
\end{lstlisting}

We define as output operand our \texttt{result} variable which will contain the
state of the LSR after the far call. \texttt{"=r"} is an operand constraint
where \texttt{"="} means that it is an output operand and \texttt{"r"} means that the
operand is a register:

\begin{lstlisting}[style=CStyle]
/* Outputs */
: "=r" (result)
\end{lstlisting}

We define as input operand the value \texttt{SERIAL\_PORT+5} which is the
address of the IO port to read. \texttt{"i"} means that it is an immediate
value:

\begin{lstlisting}[style=CStyle]
/* Inputs */
: "i" (SERIAL_PORT+5)
\end{lstlisting}

Since we have not clobbers any registers other than the output register, we can
provide an empty list:

\begin{lstlisting}[style=CStyle]
/* Clobbers */
:
\end{lstlisting}

We return \texttt{result \& 0x20} because the state of the transmitting cycle is
set on bit \texttt{5} of the LSR:

\begin{lstlisting}[style=CStyle]
return result & 0x20;
\end{lstlisting}

So this function returns \texttt{0} if the serial link is not ready or a value
other than \texttt{0} otherwise.

Now that we have a function to check if the serial link is ready to transmit, we
can write a function to print a character. In order to print a character on the
serial link, we must write to address \texttt{0x3F8} (\texttt{SERIAL\_PORT}). As
we are still in \textit{userland}, we cannot write directly to the IO port using
the \texttt{OUT} instruction. We will have to use the corresponding Pip service
which is located at index \texttt{0x30} in the GDT.

The function of the minimal partition that call the \texttt{OUT} service of Pip
to prints a character on the serial link is the following:

\begin{lstlisting}[style=CStyle]
void serial_putc(char c) {
    asm (
        "push %1;"
        "push %0;"
        "lcall $0x30, $0x0;"
        "add $0x8, %%esp"
        /* Outputs */
        :
        /* Inputs */
        : "i" (SERIAL_PORT),
          "r" ((uint32_t) c)
        /* Clobbers */
        :
    );
}
\end{lstlisting}

Now that we have these two functions, we can write our \texttt{serial\_puts}
which writes a string to the serial link:

\begin{lstlisting}[style=CStyle]
void serial_puts(const char *str) {
    for (char *it = str; *it; ++it) {
        while(!serial_transmit_ready());
        serial_putc(*it);
    }
}
\end{lstlisting}

Finally, we can print our ``Hello World'' on the serial link using our
\texttt{serial\_puts} function:

\begin{lstlisting}[style=CStyle]
void _main()
{
    const char *Hello_world_str = "Hello World !\n";
    serial_puts(Hello_world_str);
    for(;;);
}
\end{lstlisting}

\subsubsection{Linker script}

The linker script is use to specify the format and layout of the final
executable.

We start by defining the output format, which is always a flat binary, and then
the entry point of the partition which is \texttt{\_main}:

\begin{lstlisting}
OUTPUT_FORMAT(binary)
ENTRY(_main)
\end{lstlisting}

We define the mandatory \texttt{.text} section at address \texttt{0x700000},
a \texttt{.data} section for the \texttt{.data} and \texttt{.rodata} and a
\texttt{.bss} section for the \texttt{.bss}:

\begin{lstlisting}
SECTIONS {
    .text 0x700000 :
    {
        *(.text)
        . = ALIGN(0x1000);
    }

    .data :
    {
        *(.data)
        *(.rodata)
        . = ALIGN(0x1000);
    }

    .bss :
    {
        *(.bss)
    }
    end = .;
}
\end{lstlisting}

The \texttt{.text} and \texttt{.data} sections are aligned to the size of a page
using \texttt{ALIGN(0x1000)}.

\subsubsection{Makefile}

The Makefile is a file allowing to describe the steps necessary to the
generation of executables.

We start by declaring the \texttt{CFLAGS} which contains the flags used to
compile the minimal partition into intermediate objects:

\begin{lstlisting}
CFLAGS=-m32 -c -nostdlib --freestanding -fno-stack-protector -fno-pic -no-pie
\end{lstlisting}

The meaning of the flags is:

\index{dzadaz} \marginlabel{\texttt{-m32}}
Generate code for a 32-bit environment.

\marginlabel{\texttt{-c}}
Do not use the linker.

\marginlabel{\texttt{-nostdlib}}
Do not use the standard system startup files or libraries when linking.

\marginlabel{\texttt{--freestanding}}
Do not assume that standard functions have their usual definition.

\marginlabel{\texttt{-fno-stack-protector}}
Disable the stack protection.

\marginlabel{\texttt{-fno-pic}}
Disable the generation of position-independent code.

\marginlabel{\texttt{-no-pie}}
Disable the generation of position independent executable.

We then declare the \texttt{LDFLAGS} which contains the flags used to
link the minimal partition executable:

\begin{lstlisting}
LDFLAGS=-m elf_i386 -T link.ld
\end{lstlisting}

The meaning of the flags is:

\marginlabel{\texttt{-m elf\_i386}}
Create an executable that can run on \texttt{elf\_i386} processor.

\marginlabel{\texttt{-T link.ld}}
Use the linker script that we declare in the previous section.

Finally, we define some generic rules for our sources, and invoke the required
compiler for each one, calling the linker once everything has been done:

\begin{lstlisting}
CSOURCES=$(wildcard *.c)

COBJ=$(CSOURCES:.c=.o)

EXEC=minimal.bin

all: $(EXEC)
        @echo Done.

clean:
        rm -f $(COBJ) $(EXEC)

$(EXEC): $(COBJ)
        $(LD) $^ -o $@ $(LDFLAGS)

%.o: %.c
        $(CC) $(CFLAGS) $< -o $@
\end{lstlisting}

\subsection{The launcher partition}

The purpose of the launcher partition is to show how a parent partition creates
and transfers its execution flow to a child partition. To go into details, see
the source code of the launcher partition, which can be downloaded on the Pip
Protokernel website.

\subsubsection{Creating a child partition}

The first step to create a child partition is to allocate five memory pages,
using the \texttt{Pip\_AllocPage} function, for the data structures
\texttt{descChild}, \texttt{pdChild}, \texttt{shadow1Child},
\texttt{shadow2Child} and \texttt{configPagesList}:

\begin{lstlisting}[style=CStyle]
uint32_t descChild       = Pip_AllocPage();
uint32_t pdChild         = Pip_AllocPage();
uint32_t shadow1Child    = Pip_AllocPage();
uint32_t shadow2Child    = Pip_AllocPage();
uint32_t configPagesList = Pip_AllocPage();
\end{lstlisting}

For more information about these data structure, please read the PipInternals.md
file.

We ask Pip to create a child partition using the \texttt{Pip\_CreatePartition}
function, providing the previous five memory pages as arguments:

\begin{lstlisting}[style=CStyle]
Pip_CreatePartition(descChild, pdChild,
    shadow1Child, shadow2Child, configPagesList);
\end{lstlisting}

Once the child partition has been created, we need to map, using the
\texttt{Pip\_MapPageWrapper} function, each page of the child partition image,
starting with the one at the \texttt{base} address, into the virtual memory of
the newly created partition, starting at the \texttt{loadAddress} address:

\begin{lstlisting}[style=CStyle]
for (uint32_t offset = 0; offset < size; offset += PAGE_SIZE)
{
    map_page_rcode = Pip_MapPageWrapper(base + offset, descChild, loadAddress + offset);
    /* Error handling */
}
\end{lstlisting}

When all pages have been mapped, we need to allocate a memory page for the stack
of the child partition:

\begin{lstlisting}[style=CStyle]
uint32_t stackPage = Pip_AllocPage();
\end{lstlisting}

It is now necessary to create a context for the child partition. This context
must be at the beginning of the stack. Since the stack grows downwards from the
top of the memory page, the context must be at the end of the page, at the
physical address \texttt{stackPage + PAGE\_SIZE - sizeof(user\_ctx\_t)}:

\begin{lstlisting}[style=CStyle]
user_ctx_t *contextPAddr = (user_ctx_t*) (stackPage + PAGE_SIZE - sizeof(user_ctx_t));
\end{lstlisting}

and at the virtual address \texttt{STACK\_TOP\_VADDR + PAGE\_SIZE -
sizeof( user\_ctx\_t)} where \texttt{STACK\_TOP\_VADDR} is the virtual address
where the stack will be mapped:

\begin{lstlisting}[style=CStyle]
user_ctx_t *contextVAddr = (user_ctx_t*) (STACK_TOP_VADDR + PAGE_SIZE - sizeof(user_ctx_t));
\end{lstlisting}

The \texttt{user\_ctx\_t} structure contains the following members:

\marginlabel{\texttt{valid}}
This member indicates whether the structure is valid or not.

\marginlabel{\texttt{eip}}
This member must point to the first instruction of the child.

\marginlabel{\texttt{pipflags}}
This member member indicates whether the structure wants to be in virtual
\texttt{sti} or in virtual \texttt{cli}.

\marginlabel{\texttt{eflags}}
This member member indicates the state of the context (it is forced to
\texttt{0x202}).

\marginlabel{\texttt{ebp}}
This member must point to the base address of the stack page.

\marginlabel{\texttt{esp}}
This member must point to the top of stack.

We now fill the data structure with the appropriate values:

\begin{lstlisting}[style=CStyle]
contextPAddr->valid    = 0;
contextPAddr->eip      = loadAddress;
contextPAddr->pipflags = 0;
contextPAddr->eflags   = 0x202;
contextPAddr->regs.ebp = STACK_TOP_VADDR + PAGE_SIZE;
contextPAddr->regs.esp = contextPAddr->regs.ebp - sizeof(user_ctx_t);
contextPAddr->valid    = 1;
\end{lstlisting}

Once the data structure is filled, we need to map the stack of the child
partition to the virtual address \texttt{STACK\_TOP\_VADDR}:

\begin{lstlisting}[style=CStyle]
map_page_rcode = Pip_MapPageWrapper(stackPage, descChild, STACK_TOP_VADDR);
/* Error handling */
\end{lstlisting}

We now need to allocate a new memory page for the virtual Interrupt Descriptor
Table (IDT):

\begin{lstlisting}[style=CStyle]
user_ctx_t **vidtPage = Pip_AllocPage();
\end{lstlisting}

This table allows the child partition to associate an interrupt with a handler.
Here, we register the virtual address of the context of the child partition at
address \texttt{0}, \texttt{48} and \texttt{49}:

\begin{lstlisting}[style=CStyle]
vidtPage[ 0] = contextVAddr;
vidtPage[48] = contextVAddr;
vidtPage[49] = contextVAddr;
\end{lstlisting}

Finally, we map the virtual IDT memory page to the virtual memory address
\texttt{VIDT\_VADDR}:

\begin{lstlisting}[style=CStyle]
map_page_rcode = Pip_MapPageWrapper((uint32_t) vidtPage, descChild, VIDT_VADDR);
/* Error handling */
\end{lstlisting}

\subsubsection{Yielding to a child partition}

To transfer the execution flow from a parent partition to a child partition, we
have to use the \texttt{Pip\_Yield} service. Thus, the transfer of the execution
flow from the root partition of the launcher to the child partition looks like:

\begin{lstlisting}[style=CStyle]
Pip_Yield(descChild, 0, 49, 0, 0);
\end{lstlisting}

This will save the caller context at index \texttt{49}. Then this triggers
interrupt \texttt{0} in the virtual IDT of the child partition designated by
\texttt{descChild} and loads the context that was saved at that index, which is
the child context.

\subsubsection{Handling an interruption}

The root partition of the launcher handles two interrupts which are the timer
interrupt and the keyboard interrupt.

To handle an interrupt, we need to create an interrupt handler. An interrupt
handler is simply a function that will be called if the corresponding interrupt
has been triggered. The timer interrupt handler of the root partition looks
like:

\begin{lstlisting}[style=CStyle]
void timerHandler(void)
{
	printf("A timer interruption was triggered ...\n");

	// Yield to the child partition
	doYield();

	// Should never be reached
	PANIC();
}
\end{lstlisting}

Once we have declared an interrupt handler, we need to allocate a page for the
handler stack using the \texttt{Pip\_AllocPage} service:

\begin{lstlisting}[style=CStyle]
uint32_t handlerStackAddress = Pip_AllocPage();
\end{lstlisting}

and an interruption context using the \texttt{Pip\_AllocContext} service:

\begin{lstlisting}[style=CStyle]
user_ctx_t *timerHandlerContext = Pip_AllocContext();
\end{lstlisting}

Now we need to register the level \texttt{32} interrupt, which is the timer
interrupt, with the timer handler using the \texttt{Pip\_RegisterInterrupt}
service:

\begin{lstlisting}[style=CStyle]
Pip_RegisterInterrupt(timerHandlerContext, 32, timerHandler, handlerStackAddress, 0);
\end{lstlisting}

\subsection{The nanny busy beaver partition}

The purpose of this partition is to loosely test most of Pip services. To go
into details, see the source code of the nanny busy beaver partition, which can
be downloaded on the Pip Protokernel website.

\subsubsection{Deleting a child partition}

The nanny busy beaver partition is similar to the launcher partition in that it
creates and transfers its execution flow to a child partition. The only
difference is that the partition shows how a parent partition deletes a child
partition.

Before deleting a child partition, the parent partition must remove the memory
pages given to the child partition. To do this, we must call the
\texttt{Pip\_RemoveVAddr} service:

\begin{lstlisting}[style=CStyle]
Pip_RemoveVAddr(descChild, removableVPage);
\end{lstlisting}

This service takes as argument the partition descriptor of the partition and the
address of the memory page to remove.

Once the memory pages are removed, we must ask the kernel to collect the removed
memory pages. To do this, we have to use the \texttt{Pip\_Collect} service:

\begin{lstlisting}[style=CStyle]
Pip_Collect(descChild, removedVPage);
\end{lstlisting}

This service takes as argument the partition descriptor of the partition and the
address of the removed memory page.

Finally, when all the memory pages have been recovered by the parent partition,
we can delete the child partition using the \texttt{Pip\_DeletePartition}
service:

\begin{lstlisting}[style=CStyle]
Pip_DeletePartition(descChild);
\end{lstlisting}

This service takes as argument the partition descriptor of the partition to be
deleted.

\end{document}
